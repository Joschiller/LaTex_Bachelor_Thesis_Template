%%%%%%%%%%%%%%%%%%%%%%%%%%%%%%%%%%%%%%%%
%%%%%%%%%% FILE CONFIGURATION %%%%%%%%%%
%%%%%%%%%%%%%%%%%%%%%%%%%%%%%%%%%%%%%%%%
\documentclass[a4paper]{scrartcl}
\usepackage[ngerman]{babel}
\usepackage[utf8]{inputenc}

%%%%%%%%%%%%%%%%%%%%%%%%%%%%%%%%%%%%%%%%
%%%%%%%%%%      PACKAGES      %%%%%%%%%%
%%%%%%%%%%%%%%%%%%%%%%%%%%%%%%%%%%%%%%%%
\usepackage[]{graphicx}                                                                                            % figures
\usepackage{subcaption}                                                                                            % subfigures
\usepackage[]{listings}                                                                                            % code
\usepackage[backend=biber, style=ieee, natbib=true, hyperref=true, citestyle=numeric-comp, sorting=none]{biblatex} % references (use with "\cite[post]{ref}" or "\cites[pre][post]{ref}[pre][post]{ref}..."
\usepackage[hidelinks]{hyperref}                                                                                   % links within document
\usepackage{titletoc}                                                                                              % toc
\usepackage{geometry}                                                                                              % styling
\usepackage{pdfpages}                                                                                              % pdf import
\usepackage{float}                                                                                                 % force image position
\usepackage{enumitem}                                                                                              % enumeration lists with letters
\usepackage{changes}                                                                                               % further text options
\usepackage{arydshln}                                                                                              % dashed lines
\usepackage{amsmath}                                                                                               % equation styling
\usepackage{xpatch}                                                                                                % removing empty bibliography fields

%%%%%%%%%%%%%%%%%%%%%%%%%%%%%%%%%%%%%%%%
%%%%%%%%%%   CONFIGURATION    %%%%%%%%%%
%%%%%%%%%%%%%%%%%%%%%%%%%%%%%%%%%%%%%%%%
\addbibresource{bibliography.bib}
\xpatchbibdriver{online}
	{\printtext[parens]{\usebibmacro{date}}}
	{\iffieldundef{year}
		{}
		{\printtext[parens]{\usebibmacro{date}}}}
	{}
	{\typeout{There was an error patching biblatex-ieee (specifically, ieee.bbx's @online driver)}}

\stdpunctuation % enforce correct punctuation in bibliography
\renewcommand\multicitedelim{\addsemicolon\space} % enforce correct separation of multi-sources
\setcounter{biburllcpenalty}{7000} \setcounter{biburlucpenalty}{8000} % enforce correct url linebreaks
\geometry{a4paper, top=2cm, bottom=1cm, left=2cm, right=2cm, includefoot, footskip=1cm}
\newcommand{\addOverviewSection}[2]{
	\markboth{}{} % clear old header
	\section*{#1}
	\addcontentsline{toc}{section}{\protect\numberline{}#1} % add to toc
	#2
	\newpage
}
\newcounter{romanPageCounter}

% Defines a reference point for a specific category.
% E.g., \defWithLabel{My first note}{note} will be displayed as \textit{My first note} and get the label "note:My first note".
% It is advised to define custom methods to wrap the category. E.g.: \defNote{#1}[1]{\defWithLabel{#1}{note}}
\newcommand{\defWithLabel}[2]{\textit{#1}\label{#2:#1}}
% References a previously defined reference point for a specific category.
% E.g., \refWithLabel{My first note}{note} will be displayed as \textit{My first note} that links to the label "note:My first note".
% It is advised to define custom methods to wrap the category. E.g.: \refNote{#1}[1]{\refWithLabel{#1}{note}}
\newcommand{\refWithLabel}[2]{\hyperref[#2:#1]{\textit{#1}}}

\makeatletter
\renewcommand*{\@pnumwidth}{3em} % page column width for toc
\makeatother

%%%%%%%%%%%%%%%%%%%%%%%%%%%%%%%%%%%%%%%%
%%%%%%%%%%      CONTENT       %%%%%%%%%%
%%%%%%%%%%%%%%%%%%%%%%%%%%%%%%%%%%%%%%%%
\begin{document}
	%%%%%%%%%%%%%%%%%%%%%%%%%%%%%%%%%%%%%%%%
	%%%%%%%%%%       HEADER       %%%%%%%%%%
	%%%%%%%%%%%%%%%%%%%%%%%%%%%%%%%%%%%%%%%%
	\pagenumbering{Roman}
	%%%%% TITLE PAGE %%%%%
	\begin{titlepage}
	\pagestyle{empty} % no page number on first page
	\begin{center}
		\textsc{}\\[5cm]
		\textsc{\Large Title}\\[2.5cm]
		\textsc{\Large Subtitle}\\[0.5cm]
		\textsc{Author Name(s)}\\[0.5cm]
	\end{center}
\end{titlepage}

% If "\textsc" leads to display errors, replace the four lines in the middle with an alternate title format:
% \textsc{}\\[5cm]
% {\LARGE Title}\\[2.5cm]
% {\Huge Subtitle}\\[0.5cm]
% {\LARGE Author Name(s)}\\[0.5cm]

	\newpage
	\startcontents
	%%%%% CONTENT OVERVIEW SECTIONS %%%%%
	\addOverviewSection{Inhaltsverzeichnis}{\printcontents{}{1}{}}
	\addOverviewSection{Abbildungsverzeichnis}{                                     % TODO: remove the next four lines, if not needed
		\renewcommand{\listfigurename}{}
		\listoffigures
	}
	\addOverviewSection{Abkürzungsverzeichnis}{\newcommand{\defAbbreviation}[2]{
	\defWithLabel{#1}{abbrev} & #2 \\
}
\newcommand{\refAbbreviation}[1]{\refWithLabel{#1}{abbrev}}

\begin{tabular}{ll} % TODO: if there are some longer abbreviations that span over several lines, use `p{3cm}p{12.5cm}' instead of `ll'
	\defAbbreviation{...}{...}
\end{tabular}
}  % TODO: remove this line, if not needed
	\addOverviewSection{Tabellenverzeichnis}{                                       % TODO: remove the next four lines, if not needed
		\renewcommand{\listtablename}{}
		\listoftables
	}
	\addOverviewSection{Vorwort}{\paragraph{Gender-Disclaimer:} Zur besseren Lesbarkeit werden in diesem Dokument ohne Diskriminierungsabsicht Begriffe wie ``Antragsteller'' verwendet, womit Personen jeglicher Geschlechterkategorie gemeint sind.\\[2cm]

\paragraph{Eidesstattliche Erklärung:} Ich erkläre hiermit, dass ich die vorliegende Arbeit selbstständig, ausschließlich unter Verwendung der angegebenen Quellen und Hilfsmittel angefertigt habe. Aus fremden Quellen übernommene Gedanken sind als solche kenntlich gemacht. Die Arbeit wurde bisher in keiner Form einer anderen Prüfungskommission vorgelegt oder veröffentlicht.\\[1cm]

\begin{table}[H]
	\centering
	\begin{tabular}{p{5cm}p{5cm}r}
		\hline
		Ort, Datum & \hspace*{5cm} & Unterschrift
	\end{tabular}
\end{table}
}
	\setcounter{romanPageCounter}{\arabic{page}} % save roman page counter
	\pagenumbering{arabic}
	%%%%%%%%%%%%%%%%%%%%%%%%%%%%%%%%%%%%%%%%
	%%%%%%%%%%      CONTENT       %%%%%%%%%%
	%%%%%%%%%%%%%%%%%%%%%%%%%%%%%%%%%%%%%%%%
	\newcommand{\refToAppendix}[1]{\hyperref[appendix:#1]{Anhang #1}}

\section{Introduction}\label{sec:introduction}

\subsection{...}\label{sec:introduction:...}

...

% TODO: add further content sections here

	\newpage
	\markboth{}{} % clear old header
	%%%%%%%%%%%%%%%%%%%%%%%%%%%%%%%%%%%%%%%%
	%%%%%%%%%%      APPENDIX      %%%%%%%%%%
	%%%%%%%%%%%%%%%%%%%%%%%%%%%%%%%%%%%%%%%%
	\pagenumbering{Roman}
	\setcounter{page}{\theromanPageCounter}  % restore roman page counter
	%%%%% REFERENCES %%%%%
	\setlength{\emergencystretch}{3em} % enforce correct date linebreaks - this cannot be introduced earlier as it may break other line breaks
	\addOverviewSection{Literaturverzeichnis}{\printbibliography[heading=none]}
	% TODO: remove all following lines except "\stopcontents" and "\end{document}" if no appendix is needed
	%%%%% APPENDIX %%%%%
	\addOverviewSection{Anhang}{
		\startcontents[Anhang]
		\printcontents[Anhang]{}{2}{}
	}
	\markboth{}{} % clear old header
	\pagenumbering{arabic}
	\stopcontents
	%%%%%%%%%%%%%%%%%%%%%%%%%%%%%%%%%%%%%%%%
%%%%%%%%%%   CONFIGURATION    %%%%%%%%%%
%%%%%%%%%%%%%%%%%%%%%%%%%%%%%%%%%%%%%%%%
\newcommand{\startAppendixSection}[2]{
	\subsection*{{#1} {#2}}\label{appendix:#1}
	\addcontentsline{toc}{subsection}{\protect\numberline{}{{#1} {#2}}} % add to toc
}

% Use this command to include a pdf file within an appendix:
% \includeFormattedPdf{Path to PDF.pdf}{startPage-endPage}
% to include all pages use "-"
\newcommand{\includeFormattedPdf}[2]{
	\includepdf[pages={#2},pagecommand={},width=0.75\textheight]{#1}
}

% Use this command to include a single pdf page as an appendix:
% \includeFormattedPdfAppendixWithSinglePage{Appendix Symbol}{Appendix Title}{Path to PDF.pdf}
\newcommand{\includeFormattedPdfAppendixWithSinglePage}[3]{
	\includepdf[pages=1,pagecommand={\startAppendixSection{#1}{#2}},width=0.75\textheight]{#3}
}

% Use this command to include a whole pdf file as an appendix:
% \includeFormattedPdfAppendix{Appendix Symbol}{Appendix Title}{Path to PDF.pdf}
\newcommand{\includeFormattedPdfAppendix}[3]{
	\includepdf[pages=1,pagecommand={\startAppendixSection{#1}{#2}},width=0.75\textheight]{#3}
	\includepdf[pages=2-,pagecommand={},width=0.75\textheight]{#3}
}

%%%%%%%%%%%%%%%%%%%%%%%%%%%%%%%%%%%%%%%%
%%%%%%%%%%      CONTENT       %%%%%%%%%%
%%%%%%%%%%%%%%%%%%%%%%%%%%%%%%%%%%%%%%%%
\startAppendixSection{A}{... Title of the Section ...}
...

% to include pdf-files in the appendix, use:
% \includeFormattedPdf{assets/... .pdf}{-}
% or to select a range of pages
% \includeFormattedPdf{assets/... .pdf}{1-2}

\newpage
% TODO: insert further appendix sections here by duplicating the three above lines
% NOTE: To insert an appendix that solely includes a pdf, do not use all three of the above lines but instead utilize \includeFormattedPdfAppendixWithSinglePage or \includeFormattedPdfAppendix

	\stopcontents[Anhang]
\end{document}
